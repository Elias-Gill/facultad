\section{Conclusiones y Aprendizajes}

La participación en el Congreso Paraguayo de Informática 2025 resultó una experiencia académica
sumamente enriquecedora que permitió dimensionar la amplitud y complejidad actual del campo de
la gestión de centros de cómputo. Lejos de ser una disciplina estática, se pudo constatar cómo
esta área evoluciona constantemente, integrando nuevas tecnologías y metodologías que redefinen
los paradigmas tradicionales.

Fue particularmente gratificante poder presenciar cómo los conceptos teóricos vistos en clase
se materializaban en aplicaciones reales y soluciones concretas desarrolladas por profesionales
e investigadores paraguayos. El entusiasmo y la pasión que transmitían los expositores al
compartir sus trabajos hicieron que cada sesión fuera no solo educativa. Disfruté especialmente
el ambiente de colaboración y el intercambio de ideas entre participantes de diferentes
instituciones.

Más allá de las tecnologías específicas presentadas, el valor principal de esta experiencia
radica en haber comprendido la interconexión entre diversos dominios tecnológicos y su impacto
colectivo en la GCC. La convergencia de cloud computing, inteligencia artificial, IoT y
ciberseguridad ya no es una proyección futura, sino una realidad operativa que los gestores
deben aprender a administrar de manera integral.

La experiencia en general superó mis expectativas, no solo por la calidad académica de las
presentaciones, sino también por la oportunidad de conversar con distintos expositores y el
descubrimiento de las diversas líneas de investigación que se están desarrollando en el país.
Este congreso no solo amplió mi comprensión técnica, sino que también fortaleció mi motivación
para seguir profundizando en este campo.
