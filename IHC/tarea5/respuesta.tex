\documentclass[12pt, a4paper]{article}
\usepackage[spanish]{babel}
\usepackage[utf8]{inputenc}
\usepackage{graphicx}
\usepackage{hyperref}
\usepackage{geometry}
\usepackage{xcolor}
\usepackage{enumitem}
\usepackage{float}
\usepackage{pifont}
\usepackage{tabularx}
\usepackage{amsmath}
\usepackage{amssymb}
\geometry{margin=2cm}

\title{\textbf{Ejercicio de Diseño de Interfaz (UI)}}
\author{Elias Sebastian Gill Quintana}
\date{}

\begin{document}

\maketitle

\section*{Preguntas Teóricas}

\textbf{1. Define qué es el diseño de interfaz de usuario (UI) y menciona su importancia en la experiencia del usuario.}  

El diseño de interfaz de usuario (UI) es la disciplina que se encarga de crear la parte visual
e interactiva de un sistema, es decir, cómo el usuario percibe y manipula la aplicación. Su
importancia radica en que una buena interfaz facilita que el usuario logre sus objetivos de
manera rápida, eficiente y sin frustraciones, lo que impacta directamente en la experiencia de
usuario (UX).  

\textbf{2. Explica el principio de consistencia en el diseño de UI. ¿Por qué es importante?}  

El principio de consistencia en el diseño de UI se refiere a mantener uniformidad en elementos
como colores, tipografías, íconos, botones o interacciones. Es importante porque reduce la
confusión del usuario y le permite aprender más rápido cómo usar el sistema, ya que lo que
aprende en una parte se puede aplicar en otra.  

\textbf{3. ¿Qué es la retroalimentación (feedback) en el diseño de interfaces? Da un ejemplo de cómo se implementa en una página web.}  

La retroalimentación en el diseño de interfaces es la respuesta que el sistema da a las
acciones del usuario. Esto ayuda a que la persona sepa que el sistema recibió su acción y está
procesando algo. Un ejemplo en una página web es cuando el usuario envía un formulario y
aparece un mensaje que dice “Formulario enviado con éxito” o bien un ícono de carga que indica
que se está procesando la información.  

\textbf{4. ¿Qué se busca evitar con el principio de prevenir errores en el diseño de UI? Da un ejemplo de una estrategia para lograrlo.}  

El principio de prevenir errores busca evitar que el usuario cometa equivocaciones que afecten
su experiencia o el uso del sistema. Esto es importante porque los errores generan frustración
y pérdida de tiempo. Una estrategia para lograrlo es usar validaciones en los formularios, por
ejemplo, impedir que un usuario ingrese un correo sin el símbolo “@” antes de permitir que lo
envíe.  

\textbf{5. ¿Cómo ayuda la jerarquía visual en la navegación de una interfaz? Da un ejemplo de cómo se podría aplicar.}  

La jerarquía visual en una interfaz ayuda a guiar la atención del usuario hacia lo más
importante primero, organizando la información de forma clara. Esto se logra usando tamaños de
texto, colores o posiciones estratégicas. Un ejemplo sería colocar el botón principal de
“Comprar ahora” en un color más llamativo y con mayor tamaño que otros botones secundarios como
“Ver más tarde”.  

\textbf{6. ¿Por qué es importante reducir la carga cognitiva del usuario en una interfaz? Da un ejemplo.}  

Reducir la carga cognitiva es importante porque si el usuario debe pensar demasiado o procesar
mucha información al mismo tiempo, se cansa y abandona la tarea. Una interfaz clara y sencilla
permite que se enfoque en sus objetivos sin distracciones. Un ejemplo sería usar menús simples
con pocas opciones visibles y agrupar funciones en categorías para que la navegación sea más
intuitiva.  

\end{document}
