\documentclass[12pt, a4paper]{article}
\usepackage[spanish]{babel}
\usepackage[utf8]{inputenc}
\usepackage{graphicx}
\usepackage{hyperref}
\usepackage{geometry}
\usepackage{xcolor}
\usepackage{enumitem}
\usepackage{float}
\usepackage{pifont}
\usepackage{tabularx}
\geometry{margin=2cm}

\title{\textbf{Especificaciones}}
\author{
    PoliPlanner - Web
}
\date{}

\newcommand{\checkbox}{\fbox{\rule{0pt}{1.5ex}\rule{1.5ex}{0pt}}}

\begin{document}

\maketitle

\subsubsection*{\textbf{¿Qué es DevOps?}}
DevOps es presentado en el texto no como una herramienta específica ni como una metodología rígida, sino como una forma de
trabajar donde las áreas de desarrollo y operaciones dejan de actuar separadas y comienzan a colaborar de manera conjunta. La
idea central es que la entrega de software sea más rápida, confiable y continua, integrando la automatización como un apoyo,
pero con un cambio cultural más profundo donde las personas y los procesos se organizan de otra manera para lograr mejores
resultados.

\subsubsection*{\textbf{¿Las finanzas son un amigo o un enemigo?}}
El texto señala que, en un principio, el área financiera suele ser vista como un obstáculo porque requiere justificar costos
y beneficios. Sin embargo, cuando se demuestra que adoptar DevOps no es un gasto innecesario, sino una inversión con
resultados visibles en ahorro de tiempo, reducción de errores y mejora en la calidad del producto, las finanzas pasan a
convertirse en un aliado clave para impulsar la transformación en la organización.

\subsubsection*{\textbf{¿Cómo saber cuál es el puntaje?}}
Según el texto, es necesario medir el impacto real de la implementación de DevOps. No se trata únicamente de instalar nuevas
herramientas, sino de comprobar con datos concretos si la calidad del software mejora, si se reducen los fallos en producción
y si los tiempos de entrega son cada vez menores. De esta forma, el puntaje sirve como una manera objetiva de verificar que
el esfuerzo realizado está generando los resultados esperados.

\subsubsection*{\textbf{¿Qué es un producto?}}
La visión de producto en el contexto de DevOps es más amplia que solo el código o la aplicación final. Se entiende como todo
el conjunto que permite que el software funcione y llegue correctamente al usuario. Esto incluye la infraestructura, los
procesos de prueba, los mecanismos de despliegue, la seguridad y la experiencia del usuario. Así, los equipos dejan de pensar
en piezas separadas y empiezan a trabajar en soluciones integrales.

\subsubsection*{\textbf{¿Esto es más grande que DevOps?}}
El texto reconoce que sí, ya que implementar DevOps no solo introduce cambios tecnológicos, sino también culturales. Afecta
la forma en que las personas trabajan, se comunican y toman decisiones. Por esta razón, la adopción de DevOps termina
influyendo en toda la organización y no únicamente en los departamentos técnicos, mostrando que el alcance del cambio es
mucho mayor de lo que podría parecer en un principio.

\end{document}
